\documentclass{article}
\usepackage[utf8]{inputenc}
\usepackage{feynmf}
\usepackage{subfigure} 
\usepackage{caption}
\usepackage[nottoc]{tocbibind}

\title{Computing Methods in HEP: Exercise 1.}
\author{Marysabel Mejia Polania.}
\date{March 2020}

\begin{document}
\maketitle

1. Create a LaTeX document which contains Feynman graphs for the lowest order contributions to electron-positron annihilation. Place the two figures in parallel and use a joint caption below the figures. Add reference using BibTeX.

\vspace{20pt}

\begin{figure}[h]
\centering
    \subfigure{
        \begin{fmffile}{pic1}
        \begin{fmfgraph*}(150,100)
        \fmfleft{i1,i2}
        \fmfright{o1,o2}
        \fmf{electron}{i1,v1,i2}
        \fmf{photon,label=$\gamma$}{v1,v2}
        \fmf{electron}{o2,v2,o1}
        \fmflabel{$e^{-}$}{i1}
        \fmflabel{$e^{+}$}{i2}
        \fmflabel{$\mu^{-}$}{o1}
        \fmflabel{$\mu^{+}$}{o2}
        \end{fmfgraph*}
        \end{fmffile}
    }
    \subfigure{
        \begin{fmffile}{pic2}
        \begin{fmfgraph*}(150,100)
        \fmfleft{i1,i2}
        \fmfright{o1,o2,o3}
        
        \fmf{electron}{i1,v1,i2}
        \fmf{photon,label=$\gamma$}{v1,v4}
        
        \fmf{vanilla}{o1,v3}
        \fmf{electron}{v5,v4,v3}
        \fmf{photon}{v5,v6,v7,v8,v9,v3}
        \fmf{vanilla}{v5,o3}
        
        \fmflabel{$e^{-}$}{i1}
        \fmflabel{$e^{+}$}{i2}
        \fmflabel{$\mu^{-}$}{o1}
        \fmflabel{$\mu^{+}$}{o3}

        \end{fmfgraph*}
        \end{fmffile}
    }
\caption{Feynman graphs for the lowest order contributions to electron-positron annihilation. \cite{feynman}} 
\end{figure}

\bibliographystyle{unsrt}
\bibliography{ref}

\end{document}
